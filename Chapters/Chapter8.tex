% Chapter 8

\chapter{Conclusiones} % Main chapter title

\label{ch:Chapter8} % For referencing the chapter elsewhere, use \ref{Chapter1} 

\lhead{Capítulo 8. \emph{Conclusiones}} % This is for the header on each page - perhaps a shortened title

%----------------------------------------------------------------------------------------
En este trabajo se propone Pytos, un framework para emplear \textit{offloading} en entornos \textit{cloudlet}, lo cual permitirá ahorrar 
energía y optimizar el rendimiento de aplicaciones móviles que consuman abundantes recursos. Nuestro enfoque lleva la computación al borde de 
la red y visiona solucionar una de las principales limitaciones de los frameworks basados en nube: la demora de las redes WAN. 

La arquitectura propuesta es transparente para los usuarios finales, ya que descubre los \textit{cloudlets} disponibles y decide si
el \textit{offloading} es factible , o no, de manera automática. A los desarrolladores, nuestro modelo permitirá una integración simple y una 
migración de código de grano fino usando decoradores de Python. Como fue demostrado en las evaluaciones empíricas, la migración de cómputo 
a los \textit{cloudlets} puede conservar energía hasta un factor de 20.


%Furthermore, before sending computation to cloudlet mobile-context characteristics are taken in consideration (network state and history-based execution time). 
Además, la arquitectura propuesta puede ser ampliamente usada. Cada aplicación que precise de recursos adicionales (codificación de video, procesamiento
de señales, juegos, etc. ) puede ser asistida la computación en el borde de la red. Sin embargo, en el modelo propuesto, el desarrollador 
escoger la tarea conveniente que pueda ser migrada sin efectos secundarios en la aplicación. Por ejemplo, un método no puede ser migrado si este 
requiere llamadas al sistema operativo, como lecturas de sensores, uso de instrucciones de Entrada/Salida, o consultas a base de datos locales. 
En versiones futuras, se visiona solucionar tales limitaciones.

La arquitectura Pytos, si es desplegada cerca a entornos de red inalámbricas, puede mejorar la experiencia de usuario, ofreciendo, por ejemplo, 
un mejor servicio en áreas comerciales, como cafeterías, centros comerciales, y restaurantes. Si es desplegado ampliamente, 
los \textit{cloudlets} pueden reducir el tráfico de internet a nivel global. 

